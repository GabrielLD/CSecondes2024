\documentclass[12pt,a4paper]{exam}
\usepackage[utf8]{inputenc}
\usepackage[french]{babel}
\usepackage[T1]{fontenc}
\usepackage{amsmath}
\usepackage{amsfonts}
\usepackage{amssymb}
\usepackage{calculator}
\usepackage{calculus}
% \usepackage{pythontex}

\printanswers
\begin{document}

\section*{Exercice - 7 page 262}

La durée nécessaire pour qu'un son parcoure la distance $d=140~\rm m$ est $\Delta = 0,42~\rm s$ dans l'air.

\begin{questions}
    \question Calculer la célérité de l'onde sonore.

    \begin{solution}
        Par définition la vitesse est le rapport entre la distance parcourue $d$ et le temps que met l'onde a parcourir cette distance $\Delta t$:
        \[v=\dfrac{d}{\Delta t}\]

        A.N. : $v = \dfrac{140}{0,42} = $
    \end{solution}
\end{questions}
\end{document}
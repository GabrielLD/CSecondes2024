\documentclass{scrartcl}
\usepackage[driver=dvips]{geometry}
\pagestyle{empty}
\setlength{\parindent}{0pt}
\setlength{\parskip}{0pt}
\usepackage[miktex]{pstricks}
\input{./"TP 4 Familles Chimiques et Propriétés communes"-fig1-config.tex}
\makeatletter
\pst@dima=2\psxunit
\pst@dimc=-2\psxunit
\advance\pst@dima by -\pst@dimc
\ifdim\pst@dima<0pt \pst@dima=-\pst@dima\fi
\pst@dimb=16\psyunit
\pst@dimc=0\psyunit
\advance\pst@dimb by -\pst@dimc
\ifdim\pst@dimb<0pt \pst@dimb=-\pst@dimb\fi
\geometry{paperwidth=\pst@dima, paperheight=\pst@dimb, margin=0pt}
\makeatother
\begin{document}
\begin{document}(-2,0)(2,16)
\else \ifpst@dosage@PHmetre \pspicture (-5,0)(3,5)\else \pspicture (-2,0)(2,5)\fi \fi \rput (0,3.5){\pstBallon }\ifPst@burette \rput (0,4.5){\psclip {\pscustom [linestyle=none]{\pst@burette@corps }} \psframe *[linecolor=\psk@Burette@CouleurReactif ](-1,-1)(! 1 \the \pst@cnta \space 0.32 mul 2.68 add) \endpsclip } \fi \psset {linewidth=0.053}\ifPst@burette \rput (0,4.5){\pst@burette@corps \pst@burette@robinet \pst@burette@graduation }\fi \ifpst@dosage@PHmetre \rput (-3,0){\pst@dosage@pHmetre }\fi \ifPst@Agitateur@Magnetique \rput (0,1.5){\pst@dosage@aimant }\rput (0,0){\pst@dosage@support }\else \psframe (-1.5,0)(1.5,1.5)\fi \ifPst@burette \uput [0](2,11){\psk@Reactif@Burette }\fi \uput [0](2,3.5){\psk@Reactif@Becher }\endpspicture \end@SpecialObj \pr@endbox \pstDosage [glassType=erlen,burette=false]
% \end{pdfpic}


% \begin{figure}[ht]
% 	\centering

% \begin{pspicture}(0,0)(4.25,2.5)
% \psset{unit=0.5cm}
% \pstTubeEssais
% \end{pspicture}

% 	\caption{Schéma de l'expérience à compléter}
% \end{figure}

\section{Réactions avec les ions halogénures, Cl$^-$, Br$^-$ et I$^-$}

Les halogènes se trouvent très facilement sous la forme d'anions, les ions halogénures.

\begin{table}[ht]
	\centering

	\begin{tabular}{|c|c|c|}
		\hline
		\hspace{1cm}\textbf{Halogène} \hspace{1cm} & \hspace{1cm} \textbf{Nom de l'ion} \hspace{1cm} & \hspace{1cm} \textbf{Formule} \hspace{1cm} \\ \hline
		& & \\ \hline
		& & \\ \hline
		& & \\ \hline
	\end{tabular}
\end{table}

\begin{Protocol}{Protocole}
\begin{enumerate}

	\item Préparer \textbf{quatre} tubes à essais et y verser environ 2 mL des solutions ci-dessous:
 	\begin{itemize}
		\item tube 1: solution de bromure de potassium (K$^+ + $Br$^-$)
		\item tube 2: solution de chlorure de potassium (K$^+ + $Cl$^-$);
		\item tube 3: solution d'iodure de potassium (K$^+ + $Cl$^-$)
		\item tube 4: solution de nitrate de potassium (K$^+ + $Cl$^-$)
	\end{itemize}
	\item Ajouter dans les quatre tubes à essais quelques gouttes de nitrate d'argent (Ag$^+ +$NO$_3^-$)
\end{enumerate}
\end{Protocol}

\textbf{Travail à effectuer:}

\begin{enumerate}
	\item Mettre en \oe uvre le protocole précédent et schématiser les quatre expériences dans votre compte-rendu de TP;
	\item Noter vos observations
	\item Interpréter: identifier dans chaque cas l'ion qui réagit avec le nitrate d'argent, et identifier leur position dans le tableau périodique.
	\item Conclure: Que pouvez-vous affirmer sur des éléments chimiques qui appartiennent à la famille chimique (même colonne du tableau périodique) ?
\end{enumerate}
